\documentclass[11pt]{article}
\usepackage[margin=1in, top=15pt]{geometry}
\usepackage{amsmath, amssymb, xcolor}
\title{CS 225}
\author{Symbols and Formulas}
\date{Fall 2020}

\begin{document}
\maketitle
\hrule

\begin{center}\underline{\bf \large Logic}\end{center}
\bigskip

\underline{\bf Symbols:} $\geq \ \leq \ \neq \ \neg \ \sim \ \wedge \ \vee \ \oplus \ \equiv \ \rightarrow \ \leftrightarrow \ \square \ \exists \ \forall$
\bigskip

\begin{minipage}[c]{0.4\textwidth}
\underline{\bf Identities:} 
\begin{align*}
    &\sim (\sim p) \equiv p &&\text{Double Negation} \\
    &p \wedge \mathbb{T} \equiv p \ \ \ \ p \vee \mathbb{F} \equiv p &&\text{Identity} \\
    &p \vee \mathbb{T} \equiv \mathbb{T} \ \ \ \ p \wedge \mathbb{F} \equiv \mathbb{F} &&\text{Domination} \\
    &p \wedge p \equiv p \ \ \ \ p \vee p \equiv p &&\text{Idempotent} \\
    &p \vee q \equiv q \vee p \ \ \ \ p \wedge q \equiv q \wedge p &&\text{Commutative} \\
    &(p \vee q) \vee r \equiv p \vee (q \vee r) &&\text{Associative} \\
    &(p \wedge q) \wedge r \equiv p \wedge (q \wedge r) &&\text{Associative} \\
    &p \vee (q \wedge r) \equiv (p \vee q) \wedge (p \vee r) &&\text{Distributive} \\
    &p \wedge (q \vee r) \equiv (p \wedge q) \vee (p \wedge r) &&\text{Distributive} \\
    &\sim (p \wedge q) \equiv \sim p \vee \sim q &&\text{DeMorgan's} \\
    &\sim(p \vee q) \equiv \sim p \wedge \sim q &&\text{Demorgan's} \\
    &p \vee (p \wedge q) \equiv p &&\text{Absorption} \\
    &p \wedge (p \vee q) \equiv p &&\text{Absorption} \\
    &p \rightarrow q \equiv \sim q \rightarrow \sim p &&\text{Contrapositive} \\
    &p \oplus q \equiv q \oplus p &&\text{Contrapositive} \\
    &p \rightarrow q \equiv \sim p \vee q &&\text{Implication} \\
    &p \leftrightarrow q \equiv (p \rightarrow q) \wedge (q \rightarrow p) &&\text{Biconditional Equivalence} \\
    &(p \wedge q) \rightarrow r \equiv p \rightarrow (q \rightarrow r) &&\text{Exporation} \\
    &(p \rightarrow q) \wedge (p \rightarrow \sim q) \equiv \sim p &&\text{Absurdity} \\
    &p \vee q \equiv \sim p \rightarrow q &&\text{Alternate Implication} \\
    &p \wedge q \equiv \sim(p \rightarrow \sim q) &&\text{Alternate Implication} \\
    &\sim(p \rightarrow q) \equiv p \wedge \sim q &&\text{Alternate Implication} \\
    &\sim \ \forall \ x P(x) \equiv \ \exists \ x \sim P(x) &&\text{DeMorgan's for Quantifiers} \\
    &\sim \ \exists \ x Q(x) \equiv \ \forall \ x \sim Q(x) &&\text{DeMorgan's for Quantifiers} 
\end{align*}
\end{minipage}
{\color{lightgray}\vline}
\begin{minipage}[t][3.1in][b]{0.4 \textwidth}
A conditional statement $p \rightarrow q$ can also be read as:
\begin{itemize}
    \item If $p$ then $q$ 
    \item $p$ implies $q$
    \item If $p$, $q$
    \item $p$ only if $q$
    \item $q$ if $p$
    \item $q$ unless $\sim p$
    \item $q$ when $p$
    \item $q$ whenever $p$
    \item $q$ follows from $p$
    \item $p$ is a sufficient condition for $q$ ($p$ is sufficient for $q$)
    \item $q$ is a necessary condition for $p$ ($q$ is necessary for $p$)
\end{itemize}

\bigskip

\underline{\bf Proofs:}
\begin{itemize}
    \item Direct: Assume $P$ and prove $Q$.
    \item Contrapositive: Assume Not $Q$ and prove Not $P$.
    \item Contradiction: Assume $P$ and Not $Q$ and prove a contradiction.
    \item Induction: Prove base(s), assume $P(m)$, prove $P(m+1)$.
\end{itemize}
\end{minipage}

\newpage

\begin{center}\underline{\bf \large Sets}\end{center}
\bigskip

\underline{\bf Symbols:} $\in \ \not\in \ \subseteq \ \subset \ \supseteq \ \supset \ \varnothing \ \cup \ \cap \ \times$
\bigskip

\underline{\bf Common Sets:} 
\begin{align*}
    &\mathbb{N} = \{0, 1, 2, 3, \dots\}  &&\text{natural numbers} \\
    &\mathbb{Z} = \{\dots, -3, -2, -1, 0, 1, 2, 3, \dots \}  &&\text{integers ($\mathbb{Z}$ for German Zahlen, meaning "integers")} \\
    &\mathbb{Z^+} = \{1,2,3,\dots\}  &&\text{positive integers} \\
    &\mathbb{Q} = \left\{ \dfrac{p}{q} \ \Big| \ p \in \mathbb{Z}, q \in \mathbb{Z}, q \neq 0\right\}  &&\text{rational numbers} \\
    &\mathbb{U} = \{*\}  &&\text{universal set}
\end{align*}
\bigskip

\underline{\bf Identities:}
\begin{align*}
    &A \cup \varnothing = A \ \ \ \ A \cap \mathbb{U} = A &&\text{Identity} \\
    &A \cup \mathbb{U} = \mathbb{U} \ \ \ \ A \cap \mathbb{\varnothing} = \varnothing &&\text{Domination} \\
    &A \cup A = A \ \ \ \ A \cap A = A &&\text{Idempotent} \\
    &A \cup A^c = \mathbb{U} \ \ \ \ A \cap A^c = \varnothing &&\text{Complement} \\
    &A \cup B = B \cup A \ \ \ \ A \cap B = B \cap A &&\text{Commutative} \\
    &(A \cup B ) \cup C = A \cup (B \cup C) &&\text{Associative} \\
    &(A \cap B) \cap C = A \cap (B \cap C) &&\text{Associative} \\
    &A \cap (B \cup C) = (A \cap B) \cup (A \cap C) &&\text{Distributive} \\
    &A \cup (B \cap C) = (A \cup B) \cap (A \cup C) &&\text{Distributive} \\
    &(A \cup B)^c = A^c \cap B^c &&\text{DeMorgan's} \\
    &(A \cap B)^c = A^c \cup B^c &&\text{Demorgan's} \\
    &A \cup (A \cap B) = A &&\text{Absorption} \\
    &A \cap (A \cup B) = A &&\text{Absorption}
\end{align*}


\begin{center}\underline{\bf \large Series and Sums}\end{center}
\bigskip

\underline{\bf Symbols:} $\sum \ \cdot$
\bigskip

\underline{\bf Sum equations:} 
\begin{align*}
    &\text{Geometric Progression} &&a_k = a \cdot r^k &&\sum_{k=0}^n a_k = a\cdot \sum_{k=0}^n r^k = a \cdot \dfrac{r^{n+1} - 1}{r-1} \\
    &\text{Arithmetic Progression} &&a_k = a + d\cdot k &&\sum_{k=1}^n k = \dfrac{1}{2}(n^2 + n)
\end{align*}


\newpage
\begin{center}\underline{\bf \large Counting}\end{center}
\bigskip

\underline{\bf Symbols:} $\lambda$
\bigskip

\underline{\bf Equations:}
\begin{itemize}
    \item Sum Rule: 
    
    if $S$ is the union of $m$ disjoint sets $S_1, S_2, \dots, S_m$, then the number of elements in $S$ is: 
$$ S = |S_1| + |S_2| + \dots + |S_m| $$
    \item Product Rule: 
    
    for a sequence of $m$ choices the total number of elements is: 
$$ P(m) = |P_1|\cdot |P_2| \cdot \dots \cdot |P_m| $$
    \item Subset Exclusion: 
    
    if $C = A - B$ is the difference of two sets, where $B \subseteq A$, then the number of elements in $C$ is:
$$ |C| = |A| - |B| $$
    \item Inclusion/Exclusion: 
    
    when a set $D$ to be counted is the union of non-disjoint sets $A$ and $B$, the number of elements in $D$ is:
$$ |D| = |A| + |B| - |A\cap B| $$
\end{itemize}
\bigskip

\underline{\bf Permutations and Combinations:}
\begin{itemize}
    \item $r$-permutations: $P(n,r) = \dfrac{n!}{(n-r)!}$
    \item $r$-permutations with repetition allowed: $n^r$
    \item The number of different permutations of $n$ objects, where there are $n_1$
indistinguishable objects of type 1, and $n_2$ indistinguishable objects
of type 2, \dots and $n_k$ indistinguishable objects of type $k$ is $\dfrac{n!}{n_1! n_2! \dots n_k!}$
    \item $r$-combinations $C(n,r) = \dfrac{n!}{r!(n-r)!}$
    \item $r$-combinations with repetition allowed: $C(n+r-1,r) = \dfrac{(n+r-1)!}{r!(n+r-1-r)!} = \dfrac{(n+r-1)!}{r!(n-1)!}$
\end{itemize}

\end{document}